\documentclass[aspectratio=169]{beamer}

% \usetheme{Frankfurt}
\usetheme{Warsaw}

\usecolortheme{seahorse}
\usecolortheme{rose}

\setbeamertemplate{footline}[frame number]
\setbeamertemplate{navigation symbols}[vertical]

% ****************************************************************
% Nikhil's defs

% ----------------
% EMPTY BOXES OF VARIOUS WIDTHS, FOR INDENTATION
\newcommand{\hm}{\hspace*{1em}}
\newcommand{\hmm}{\hspace*{2em}}
\newcommand{\hmmm}{\hspace*{3em}}
\newcommand{\hmmmm}{\hspace*{4em}}

\newcommand{\tinytt}{\tiny\tt}
\newcommand{\scripttt}{\scriptsize\tt}

% ****************************************************************

\title{A Tour of the RISC-V ISA Formal Specification}

\author{RISC-V Foundation ISA Formal Spec Technical Committee}

\date{At RISC-V Summit, San Jose \\ December 12, 2019}

% ****************************************************************
% ****************************************************************
% ****************************************************************

\begin{document}

% ----------------------------------------------------------------

\begin{frame}
  \titlepage
\end{frame}

\begin{frame}
  \frametitle{Abstract}

  \scriptsize

  In this hands-on tutorial, we would like to familiarize a broad
  RISC-V community with the RISC-V ISA Formal Specification, with a
  view to encouraging its use on a daily basis.  We hope to cover the
  following: (1) A reading tour of the spec, so people are
  subsequently able to consult the spec on their own; (2) How to build
  and execute the spec on RISC-V binaries (from ISA tests to operating
  systems) to produce reference executions; (3) How to build and
  execute the spec on the Compliance Suite.  If there is time and
  interest, additional topics are: (4) How to extend the spec for new
  and custom ISA extensions; and (5) A description of how the spec is
  being used for formal proofs of compiler and OS correctness, and
  hardware implementation correctness.  The target audience for this
  tutorial are people who are NOT specialists in formal methods, but
  who wish to use the spec in daily practice (e.g. use cases 1-3
  above).  Attendees who can run Linux on their laptops (natively, in
  a VM, or remotely) will be able to follow the speaker in a hands-on
  manner.

\end{frame}


% ----------------------------------------------------------------

% \begin{frame}
%   \frametitle{Outline}
%   \tableofcontents
% \end{frame}

% ****************************************************************

\begin{frame}
  \frametitle{Under Construction}

  \begin{block}{Under Construction}

    \begin{itemize}

    \item The slides for this tutorial are in preparation, and will be
      made available in time for the tutorial presentation on December
      12, 2019 at the RISC-V Summit.

    \item Please prepare for the tutorial in advance by following the
      instructions in the separate slide deck {\tt
        Slides\_Installation.pdf} on how to download the spec, and
      optionally to build an executable version of the spec.

    \end{itemize}
  \end{block}

\end{frame}


% ****************************************************************

\section{Intro}

% ****************************************************************

\section{Reading the ISA to understand semantics of each RISC-V instruction}

% ****************************************************************

\section{Executing ISA and Compliance tests}

% ================================================================

\subsection{Executing standard RISC-V ISA tests}

% ================================================================

\subsection{Executing Compliance tests}

% ****************************************************************

\section{Extra material}

% ================================================================

\subsection{Extending the spec with new ISA extensions}

% ================================================================

\subsection{How the spec is being used with formal methods}

% ****************************************************************

\end{document}
